\documentclass{beamer} % Установка класса документа

\usetheme{Copenhagen} % Установка темы презентации

\colorlet{beamer@blendedblue}{green!40!black} % Установка цветовой гаммы презентации

\usepackage[T2A]{fontenc}   % Подключение внутренней кодировки LaTeX
\usepackage[utf8]{inputenc} % Подключение кодировки набираемого текста
\usepackage[russian]{babel} % Подключение особенностей обработки текста

\renewcommand{\familydefault}{\sfdefault} % Установка семейства шрифтов

\usepackage[export]{adjustbox} % Управление расположение картинок

\usepackage{geometry} % Подключение пользовательской разметки документа

\parindent = 0pt  % Установка отступа в начале абзаца
\parskip   = 0pt  % Установка интервала между абзацами
\tolerance = 2000 % Установка меры разреженности строки

\flushbottom % Установка выравнивания высоты страниц

\usepackage{amsmath}  % Подключение средств форматирования математических выражений
\usepackage{amsbsy}   % Подключение \boldsymbol и \pmb для набора полужирных символов
\usepackage{amscd}    % Подключение процедуры создания простых коммутативных диаграмм
\usepackage{amsfonts} % Подключение дополнительных шрифтов для математических символов
\usepackage{amsopn}   % Подключение \DeclareMathOperator для определения новых команд
\usepackage{amssymb}  % Подключение дополнительных математических символов и шрифтов
\usepackage{amstext}  % Подключение \text для набора текста внутри выключных формул
\usepackage{amsthm}   % Подключение окружения proof и улучшенной декларации теорем
\usepackage{amsxtra}  % Подключение дополнительных средств форматирования формул

\usepackage{latexsym} % Подключение дополнительных математических символов

\usepackage{listings} % Подключение форматирования кода
\usepackage{color}    % Подключение управления цветом
 
\definecolor{c_1}{rgb}{0.00, 0.50, 0.00} % Цвет темно-зеленый для ссылок
\definecolor{c_2}{rgb}{0.00, 0.00, 0.50} % Цвет темно-синий для листингов
\definecolor{c_3}{rgb}{0.50, 0.00, 0.00} % Цвет темно-красный для резерва

\lstset % Настройка пакета listings
{
  backgroundcolor   = \color{white},        % Установка цвета заднего плана  
  basicstyle        = \ttfamily\color{c_2}, % Установка размера и цвета шрифта
  breakatwhitespace = true,                 % Установка разрывов на пробелах
  breaklines        = true,                 % Установка переноса длинных строк
  captionpos        = none,                 % Установка позиции имени листинга
  commentstyle      = \color{c_2},          % Установка цвета комментариев кода
  deletekeywords    = {},                   % Установка удаленных ключевых слов  
  escapeinside      = {\%*}{*)},            % Установка добавления LaTeX в коде  
  extendedchars     = false,                % Установка дополнительных символов 
  frame             = L,                    % Установка типа рамки вокруг кода
  framexleftmargin  = -8pt,                 % Установка размера левого отступа
  keepspaces        = true,                 % Установка выравнивания пробелов
  keywordstyle      = \color{c_2},          % Установка цвета ключевых слов  
  language          = C++,                  % Установка языка программирования
  otherkeywords     = {},                   % Установка добавочных ключевых слов   
  numbers           = none,                 % Установка позиции нумерации строк
  numbersep         = 0pt,                  % Установка отступа нумерации строк
  numberstyle       = \color{black},        % Установка цвета нумерации строк
  showspaces        = false,                % Установка пробелов символом '_'
  showstringspaces  = false,                % Установка пробелов символом '_'
  showtabs          = false,                % Установка табуляторов видимыми
  stepnumber        = 1,                    % Установка периода нумерации строк
  stringstyle       = \color{c_2},          % Установка цвета строковых литералов
  tabsize           = 2,                    % Установка размера табуляции в коде
}

%\usepackage{hyperref} % Подключение использования гиперссылок
 
%\hypersetup % Настройка пакета hyperref
%{
%  linkcolor  = c_1,  % Установка цвета ссылок
%  urlcolor   = c_1,  % Установка цвета гиперссылок
%  colorlinks = true  % Установка отображения ссылок
%}

\usepackage{sectsty} % Подключение возможности стилизации заголовков статей

\sectionfont   {\Large\selectfont} % Установка размера шрифта подзаголовков
\subsectionfont{\large\selectfont} % Установка размера шрифта подподзаголовков

\usepackage{titlesec}  % Подключение возможности форматирования заголовков статей

\titlelabel{} % Установка формата заголовков

\usepackage{ragged2e} % Настройки титульной страницы

\title{Программная инженерия с использованием C\texttt{++}} % Название презентации

\author{Московский Физико\,-Технический Институт} % Альма-матер автора

\date{}

\begin{document}

\frame{\titlepage} % Титульный слайд

\begin{frame}{\bf Автор курса}

    \textbf{\;\;\;\;\,\,\,\,Иван Сергеевич Макаров}

    \bigskip
    
    \begin{itemize}

        \item Выпускник Московского Физико\,-Технического Института

        \item Кандидат технических наук по специальности 1.2.1

        \item Доцент Высшей Школы Программной Инженерии МФТИ
    
        \item Автор 11 докладов на конференциях IEEE и ICPS

        \item Автор 20 публикаций в рецензируемых научных изданиях
        
    \end{itemize}

    \begin{block}
    \justifying В настоящее время я занимаюсь разработкой вычислительных систем с использованием технологий параллельного програм- мирования, в частности, инфраструктурных компонент систем автоматизированной торговли для централизованных рынков.
    \end{block}
    
\end{frame}

\begin{frame}{\bf Программа курса}

{\footnotesize

    \begin{itemize}

        \setlength\itemsep{0.1em}

        \item 01. Общее введение и обзор технологий

        \item 02. Структурное программирование

        \item 03. Объектно-ориентированное программирование

        \item 04. Обобщенное программирование

        \item 05. Паттерны и технологии проектирования

        \item 06. Организация проектов и библиотек

        \item 07. Обработка ошибок и исключений

        \item 08. Особенности математических вычислений

        \item 09. Низкоуровневое управление памятью

        \item 10. Коллекции объектов и контейнеры

        \item 11. Итераторы и алгоритмы на диапазонах

        \item 12. Кодирование символов и парсинг текстов

        \item 13. Потоки ввода-вывода и сериализация

        \item 14. Параллельное программирование

        \item 15. Компьютерные сети и сетевые технологии
        
    \end{itemize}
    
}

\end{frame}

\begin{frame}{\bf Определение C\texttt{++}}

    \begin{block}
    \justifying C\texttt{++} -- это компилируемый язык программирования широкого назначения со слабой статической типизацией, который под- держивает несколько парадигм программирования и предос- тавляет как низкоуровневые, так и высокоуровневые средства.
    \end{block}

    \begin{itemize}

        \item Процессор понимает низкоуровневый машинный код

        \item Программист пишет высокоуровневый исходный код

        \item Компилятор транслирует исходный код в машинный

        \item Типы всех объектов известны во время компиляции

        \item Допустимы автоматические неявные преобразования
        
    \end{itemize}
    
\end{frame}

\begin{frame}{\bf Парадигмы программирования}
    
    \begin{itemize}

        \item Декларативное программирование -- описание результатов

        \item Императивное программирование -- последовательности

        \item Процедурное программирование -- подпрограммы

        \item Функциональное\,программирование -- композиции\,функций
        
        \item Структурное программирование -- ветвления и циклы
        
        \item Объектно-ориентированное программирование -- классы
        
        \item Обобщенное программирование -- типы и шаблоны
        
        \item Параллельное программирование -- процессы и потоки
        
        \item Событийно-ориентированное программирование -- события
        
    \end{itemize}
    
\end{frame}

\begin{frame}{\bf Формирование C\texttt{++}}
    
    \begin{itemize}

        \item Изначально разрабатывался как набор расширений C

        \item В настоящее время является самостоятельным языком

        \item Унаследовал компоненты Ada, Algol, Fortran и Simula

        \item Повлиял на Java, Go, Python и многие другие языки

        \item Занял место на рынке и имеет достойных конкурентов
         
    \end{itemize}

    \begin{block}
    \justifying Первый коммерческий релиз C\texttt{++} состоялся 14 октября 1985 г.
    \end{block}
    
\end{frame}

\begin{frame}{\bf Стандартизация C\texttt{++}}

    \begin{itemize}

        \item C\texttt{++}98 -- фундаментальный стандарт

        \item C\texttt{++}03 -- патч

        \item Technical Report 1 2007 \texttt{/} Boost

        \item C\texttt{++}11 -- ядро языка \texttt{+30}\%, стандартная библиотека \texttt{+100}\%

        \item C\texttt{++}14 -- патч

        \item C\texttt{++}17 -- патч

        \item C\texttt{++}20 -- важные дополнения

        \item C\texttt{++}23 -- патч

        \item C\texttt{++}26 -- следующий разрабатываемый стандарт
         
    \end{itemize}

    \begin{block}
    \justifying Дополнительные возможности предоставляются библиотеками.
    \end{block}
    
\end{frame}

\begin{frame}{\bf Использование C\texttt{++}}
    
    \begin{itemize}

        \item Операционные системы и системы управления

        \item Высоконагруженные системы обработки данных

        \item Математическое и физическое моделирование

        \item Финансовые системы автоматической торговли

        \item Решения для отраслей высокой ответственности
        
    \end{itemize}

    \begin{block}
    \justifying C\texttt{++} сочетает низкоуровневые и высокоуровневые технологии.
    \end{block}
    
\end{frame}

\begin{frame}{\bf Инструменты разработчика}

    \begin{itemize}

        \item Системы сборки -- Clang, GCC, ICC, MinGW, MSVC

        \item Среды разработки -- CLion, Visual Studio, XCode

        \item Редакторы кода -- Sublime, Vim, Visual Studio Code

        \item Отладчики и профилировщики -- GDB, Valgrind

        \item Системы автоматизации сборки -- CMake, MSBuild

        \item Системы контроля версий -- Git, SVN, Mercurial

        \item Сервисы для хостинга проектов -- GitHub, Bitbucket

        \item Графические клиенты -- GitHub Desktop, SmartGit

        \item Системы управления задачами -- Asana, Jira, Trello

    \end{itemize}
    
\end{frame}

\begin{frame}{\bf Полезные ресурсы}

    \begin{itemize}

        \item learncpp.com -- начальные учебные материалы

        \item cppreference.com -- справочное руководство

        \item boost.org -- документации к библиотекам Boost

        \item github.com -- сервис для хостинга проектов

        \item stackoverflow.com -- форум вопросов и ответов

    \end{itemize}

    \begin{block}
    \justifying Список изученной мной литературы доступен на \href{https://docs.google.com/spreadsheets/d/1MAx4-DoZUrZEB210XJ524e-0OogZRvH7QBorja_tGiw/edit?usp=sharing}{Google диске}.
    \end{block}
    
\end{frame}

\begin{frame}{\bf Этапы жизненного цикла проекта}

    \begin{itemize}

        \item Формирование набора требований

        \item Оценка необходимых ресурсов

        \item Разработка архитектуры решения

        \item Детальное проектирование модулей

        \item Механическое написание кода

        \item Отладка и оптимизация системы

        \item Внедрение и техническая поддержка

        \item Вывод продукта из эксплуатации

    \end{itemize}
    
\end{frame}

\end{document}