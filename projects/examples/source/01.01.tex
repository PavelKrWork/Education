\documentclass{beamer}

\usetheme{Copenhagen}

\colorlet{beamer@blendedblue}{green!40!black}



\usepackage[export]{adjustbox}
\usepackage[russian]{babel}
\usepackage[T2A]{fontenc}
\usepackage[utf8]{inputenc}



\usepackage{amsbsy}
\usepackage{amscd}
\usepackage{amsfonts}
\usepackage{amsmath}
\usepackage{amsopn}
\usepackage{amssymb}
\usepackage{amstext}
\usepackage{amsthm}
\usepackage{amsxtra}
\usepackage{array}
\usepackage{ctable}
\usepackage{geometry}
\usepackage{hyperref}
\usepackage{latexsym}
\usepackage{listings}
\usepackage{makecell}
\usepackage{ragged2e}
\usepackage{titlesec}
\usepackage{tocloft}
\usepackage{xcolor}



\renewcommand{\familydefault}{\sfdefault}

\renewcommand{\arraystretch}{1.25}

\newcolumntype{L}[1]{>{\raggedright\let\newline\\\arraybackslash\hspace{0pt}}m{#1}}

\parindent = 0pt
\parskip   = 0pt
\tolerance = 100

\flushbottom

\definecolor{B}{rgb}{0.00, 0.00, 0.50}

\lstset
{
    backgroundcolor   = \color{white},      % Установка цвета заднего плана  
    basicstyle        = \ttfamily\color{B}, % Установка размера и цвета шрифта
    breakatwhitespace = true,               % Установка разрывов на пробелах
    breaklines        = true,               % Установка переноса длинных строк
    captionpos        = none,               % Установка позиции имени листинга
    commentstyle      = \color{B},          % Установка цвета комментариев кода
    deletekeywords    = {},                 % Установка удаленных ключевых слов  
    escapeinside      = {\%*}{*)},          % Установка добавления LaTeX в коде  
    extendedchars     = false,              % Установка дополнительных символов 
    frame             = L,                  % Установка типа рамки вокруг кода
    framexleftmargin  = -8pt,               % Установка размера левого отступа
    keepspaces        = true,               % Установка выравнивания пробелов
    keywordstyle      = \color{B},          % Установка цвета ключевых слов  
    language          = C++,                % Установка языка программирования
    otherkeywords     = {},                 % Установка добавочных ключевых слов   
    numbers           = none,               % Установка позиции нумерации строк
    numbersep         = 0pt,                % Установка отступа нумерации строк
    numberstyle       = \color{black},      % Установка цвета нумерации строк
    showspaces        = false,              % Установка пробелов символом '_'
    showstringspaces  = false,              % Установка пробелов символом '_'
    showtabs          = false,              % Установка табуляторов видимыми
    stepnumber        = 1,                  % Установка периода нумерации строк
    stringstyle       = \color{B},          % Установка цвета строковых литералов
    tabsize           = 2,                  % Установка размера табуляции в коде
}



\begin{document}



\title{\bf Software Engineering} 

\author{Moscow Institute of Physics and Technology}

\date{}

\frame{\titlepage}



\begin{frame}{\bf Course Lecturer}

    \textbf{\;\;\;\;\,\,\,\,PhD Ivan Sergeevich Makarov}

    \bigskip
    
    \begin{itemize}

        \item Graduate of the Moscow Institute of Physics and Technology

        \item Candidate of Technical Sciences in specialty 1.2.1

        \item Associate Professor and Course Lecturer at the MIPT

        \item Software Developer and AI Researcher since 2010

        \item Author of 20 publications in peer\,-\,reviewed scientific journals
    
        \item Author of 11 reports at international conferences
        
    \end{itemize}

    \begin{block}
    \justifying Currently, I am engaged in the design and development of various computing systems using concurrent\:and\:network\:programming, in particular, I am working on the optimization of infrastructure com- ponents\;for\;financial\;systems\;of\;automated\;high\,-\,frequency\;trading.
    \end{block}
    
\end{frame}

\begin{frame}{\bf Course Program}

{\footnotesize

    \begin{itemize}

        \setlength\itemsep{0.1em}

        \item 01. Introduction and Brief Overview
    
        \item 02. Basics of Programming
    
        \item 03. Object\,-\,Oriented Programming
    
        \item 04. Generic Programming
    
        \item 05. Software Architecture Patterns
    
        \item 06. Projects and Libraries
    
        \item 07. Handling Errors and Debugging
    
        \item 08. Instruments of Calculus
    
        \item 09. Detailed Memory Management
    
        \item 10. Collections and Containers
    
        \item 11. Iterators and Algorithm Libraries
    
        \item 12. Text Data Processing
    
        \item 13. Streams and Data Serialization
    
        \item 14. Concurrent Programming
    
        \item 15. Network Technologies and Tools
        
    \end{itemize}
    
}

\end{frame}

\begin{frame}{\bf C\texttt{++} Definition}

    \begin{block}
    \justifying C\texttt{++}\;is\;a\;compiled\:general\,-\,purpose\:programming\:language\;based\:on weak static type system. This language supports multiple program- ming\:paradigms\:and\:provides\:both\:low\,-\,level\:and\:high\,-\,level\:features.
    \end{block}

    \begin{itemize}

        \item The processor\;understands\;only\;low\,-\,level\;machine\;code

        \item The developer writes high\,-\,level source code

        \item The compiler translates the source code into machine code

        \item The types of all objects are known at compile time

        \item Various automatic implicit type conversions are allowed
        
    \end{itemize}
    
\end{frame}

\begin{frame}{\bf C\texttt{++} Evolution}
    
    \begin{itemize}

        \item Originally\;developed as\;a\;set\;of the\;C\;language\;extensions

        \item Currently is an independent and full\,-\,fledged language

        \item Inherited components of Ada, Fortran, Simula and others

        \item Influenced\;Java,\:Go,\:Python\;and\;many\;other\;languages

        \item Has\;taken\;place\;in\;the\;market\;and\;has\;several\;competitors
         
    \end{itemize}

    \begin{block}
    \justifying The\:first\:commercial\:release\:of\:the\:C\texttt{++}\:language\:was\:on\:14\texttt{/}10\texttt{/}1985.
    \end{block}
    
\end{frame}

\begin{frame}{\bf C\texttt{++} Standards}

    \begin{itemize}

        \item C\texttt{++}98 -- fundamental\,standard

        \item C\texttt{++}03 -- patch

        \item Technical Report 1 2007 and various Boost libraries

        \item C\texttt{++}11 -- significant extensions

        \item C\texttt{++}14 -- patch

        \item C\texttt{++}17 -- patch

        \item C\texttt{++}20 -- significant extensions

        \item C\texttt{++}23 -- patch

        \item C\texttt{++}26 -- the next standard under development
         
    \end{itemize}

    \begin{block}
    \justifying Additional features are provided by libraries such as Boost and Qt.
    \end{block}
    
\end{frame}

\begin{frame}{\bf C\texttt{++} Use Cases}
    
    \begin{itemize}

        \item Operating systems and robotics software

        \item Highly loaded data processing systems

        \item Gaming software and simulation systems

        \item Financial\:systems\:of\:automated\:trading

        \item Software for highly responsible industries
        
    \end{itemize}

    \begin{block}
    \justifying The C\texttt{++} language provides both low\,-\,level and high\,-\,level features.
    \end{block}
    
\end{frame}

\begin{frame}{\bf Programming Paradigms}
    
    \begin{itemize}

        \item Declarative\;programming -- SQL and HTML

        \item Imperative\, programming -- \,statements

        \item Procedural programming -- \,subroutines

        \item Functional\, programming -- \,Lisp,\:Erlang\:and\:Haskell
        
        \item Structured\, programming -- \,sequences, selections and loops 
        
        \item Object\,-\,oriented\,programming\;--\;classes
        
        \item Generic programming -- templates

        \item Event\,-\,driven programming -- events and callbacks
        
        \item Concurrent programming -- threads,\;processes\;and\;networks
        
    \end{itemize}

    \begin{block}
    \justifying The\:same\:program\:can\:be\:implemented\:in\:many\:different\:paradigms.
    \end{block}
    
\end{frame}

\begin{frame}{\bf Instruments}

    \begin{tabular}{|l|l|l|}

        \hline

        \textbf{Instrument} & \textbf{Considered} & \textbf{Alternative} \\

        \hline
        
        \text{Linux\,operating\,system} & Ubuntu & Debian, CentOS \\

        \hline

        \text{Environment} & Visual\;Studio\;Code & CLion \\

        \hline
         
        Toolset\texttt{::}Compiler & g\texttt{++} from GCC & Clang \\

        \hline

        Toolset\texttt{::}Builder & CMake & Bazel, Ninja \\

        \hline 

        Toolset\texttt{::}Debugger & GDB, Valgrind & LLDB \\

        \hline

        Toolset\texttt{::}Profiler & Google.Benchmark & gperftools \\

        \hline

        \text{Version control system} & Git & no \\

        \hline

        \text{Git graphical client} & SmartGit, IDE & GitHub Desktop \\

        \hline

        \text{Project hosting system} & GitHub & Bitbucket \\

        \hline
        
    \end{tabular}
    
\end{frame}

\begin{frame}{\bf References}

    \begin{itemize}

        \item learncpp.com -- basic\,educational\,materials

        \item cppreference.com -- language reference

        \item boost.org -- Boost libraries documentation

        \item github.com -- open\,projects\,and\,libraries

        \item stackoverflow.com -- developers QA forum

    \end{itemize}

    \begin{block}
    \justifying The\:list\:of\:recommended\:books\:is\:available\:at\:my\:own\:\href{https://docs.google.com/spreadsheets/d/1MAx4-DoZUrZEB210XJ524e-0OogZRvH7QBorja_tGiw/edit?usp=sharing}{Google\,-\,table}.
    \end{block}
    
\end{frame}

\end{document}